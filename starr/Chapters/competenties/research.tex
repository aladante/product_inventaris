\competentie
{% competentieformulier
	\competentieformulier
	{% toelichting
		Je bent onderzoekend en brengt verschillende
		aspecten van een vraagstuk of probleem vanuit
		verschillende perspectieven in kaart. Je verzamelt
		relevante informatie uit erkende bronnen. Je
		analyseert deze informatie en brengt deze op
		systematische wijze met elkaar in verband. Op basis
		hiervan vorm je een oordeel en kom je tot een
		oplossing. Je kunt verschillende invalshoeken
		gebruiken om tot nieuwe ideeën en oplossingen te
		komen.
	}
	{% deelcompetenties
		analyse en
		oordeelsvorming,
		onderzoeken,
		creativiteit
	}
	{%
		Proof
	}
	{%
		Deze competentie wordt beoordeeld met behulp van een STARR
	}
	{% verwijzing naar bewijs
		\href{https://github.com/aladante/product_inventaris}{git repo}
	}
}
{% bewijzen
	\bewijs
	{%naam
		Netwerk verkeer deligeren binnen een cluster
	}
	{% starr
		\starr
		{% betreft
			Creativiteit,
			Analyse en
			oordeelsvorming
		}
		{% datum
			14-05-2022
		}
		{% situatie
			Voor de deployment van de applicatie gebruik ik Kubernetes.
			Binnen een cluster kan je endpoints een IP geven om zo een hostnaam te gebruiken.

		}
		{% taak
			Onderzoeken en realisatie van het netwerk verkeer binnen de cluster zodat alles op de juiste locatie aankomt.
		}
		{% activiteiten
			Ik ben begonnen met de cursus. Hier heb ik veel essentiele informatie uit gehaald over het gebruik van Kubernetes.

			De klant gaf aan dat ze het graag op een host name heeft staan om het zo makkelijker te maken voor de gebruikers, een ip onhouden is niet handig en niet praktisch.

			Ik heb het domijn \href{http://vacinfi.com/login}{http://vacinfi.com/login} gekocht voor de werkgever en ben deze gaan gebruiken om de applicatie te tonen op het WWW.

			Het probleem waar ik mee zat is dat ik niet 2 outgoing services wou hebben.
			Ik wil 1 domein hebben waar het domein aan gekoppeld is.

			Met react compile ik een efficiente versie die de pagina servert via NGINX.
			NGINX stuur op aanvraag van het domijn de juiste statische pagina's door naar de gebruiker zodat de gebruiker de applicatie kan gebruiken.

			Na veel testen en proberen van verschillende oplossingen, zoals /api te exposes naar end-point 1 en de root te exposes naar de react paginas of een indivuduele loadbalancer te gebruiken die het verkeer deligeert naar beide services ben ik tot een werkzame implentatie gekomen.

			Mijn naar mijn zeggen creatieve oplossing maakt gebruik van de front-end service.
			Het verkeer werd op deze service al gedeligeerd door NGINX.
			Bij deze service heb ik de NGINX aangepast zodat het verkeer met end-point /graphql wordt gedeligeerd naar de back-end server.

			Op deze manier is er 1 openbaar IP waar de hostnaam aan gekoppeld staat en komt al het verkeer aan bij de juiste services.

		}
		{% resultaat
			Het resultaat is een werkende applicatie die draait binnen een cluster waar al het intern verkeer correct gedeligeerd wordt naar de juiste services.
		}
		{% reflectie
			Door te onderzoeken naar verschillende opties om netwerk verkeer binnen een cluster te deligeren naar de juiste service ben ik tot een correcte creatieve oplossing gekommen.

			Ik moet nog ssl certificaten generen voor de applicatie zodat er SSL gebruikt wordt.
			Dit is een stuk makkelijker nu er 1 centrale plek is waar het verkeer aankomt.
			Op deze manier hoef ik maar 1 certificaat te onderhouden in plaats van verschillende als ik bijvoorbeeld 2 IP's had exposed naar de buiten wereld.

			Door te onderzoeken en analyseren ben ik tot verschillende opties gekozen waar ik uiteindelijk de voor mijn situatie beste oplossing heb gekozen.

		}
		{
			The code of the application.
		}
	}
	{% bewijs
		\href{https://github.com/aladante/product_inventaris}{git repo}
		\href{https://github.com/aladante/product_inventaris/blob/main/front-end/nginx/nginx.conf}{nginx config}
	},
}
% section section name (end)
